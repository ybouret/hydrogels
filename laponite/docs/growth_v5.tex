\documentclass[11pt]{revtex4}
\usepackage{graphicx}
\usepackage{amssymb,amsmath}
\usepackage{epstopdf}
\usepackage{bm}
\DeclareGraphicsRule{.tif}{png}{.png}{`convert #1 `dirname #1`/`basename #1 .tif`.png}


%\date{}                                           % Activate to display a given date or no date



\begin{document}
\title{Let's grow one bubble}

\maketitle

\section{Description}
\subsection{Bubble}
We assume that a bubble with a volume $V$ growths with a pressure $P$ and a number of
gaseous mol $N$.
We want to find the growth rate under the assumption that the bubble is formed with dissolved gaz
and that the gas must diffuse through the expanding fluid.\\
Since this can be in 2 or 3 dimensions, we give the bubble a radius $R$.
We set $\alpha=1$ for 2D (and the bubble has a height $e$), and $\alpha=3$ for $3D$.

\subsection{Velocity fluid}
The incompressible fluid velocity is assume to be
\begin{equation}
	\vec{v}(r,t) = \dot{R} \left(\dfrac{R}{r}\right)^\alpha \vec{u}_r
\end{equation}

\subsection{Matter exchange}
In the following, the $\star$ exponent is for the contact value, at $r=R(t)$.\\
In 2D
\begin{equation}
	\left[\partial_t N\right]_{2d} = 2\pi e R  D \partial_r C^\star.
\end{equation}
In 3D
\begin{equation}
	\left[\partial_t N\right]_{3d} =  4\pi R^2 D \partial_r C^\star.
\end{equation}


\subsection{Matter transport}
The incompressible diffusion equation is
\begin{equation}
	\partial_t C + \dot{R} \left(\dfrac{R}{r}\right)^\alpha  \partial_r C = D \left( \partial_{rr} C + \dfrac{\alpha}{r} \partial_r C \right) = D \, \dfrac{1}{r^\alpha} \partial_r \left(r^\alpha\partial_r C\right)
\end{equation}

\section{Matter Intake}
\subsection{Generic case}
We use the scaled variable
\begin{equation}
	s(r,t)
\end{equation}
and
\begin{equation}
	C(r,t) = C^\star + \left[C_\infty-C^\star\right] Q(s) = C^\star + \delta C Q(s)
\end{equation}
so that
\begin{equation}
	\partial_r C = \delta C \left(\partial_r s\right) Q'(s)
\end{equation}
and
\begin{equation}
	\partial_{rr} C = \delta C \left[ \left(\partial_r s\right)^2 Q'' + \left(\partial_{rr}s\right) Q'\right]
\end{equation}
and
\begin{equation}
	\partial_{t} C = \dot{C}^\star \left[1-Q\right] + \delta C \left(\partial_t s\right) Q'
\end{equation}

We obtain
\begin{equation}
	\dfrac{\dot{C}^\star}{\delta C} \left[1-Q\right] +  \left(\partial_t s\right) Q' 
	+\dot{R} \left(\dfrac{R}{r}\right)^\alpha \left(\partial_r s\right) Q' = D  \left[ \left(\partial_r s\right)^2 Q'' + \left(\partial_{rr}s\right) Q' + \dfrac{\alpha}{r} \left(\partial_r s\right) Q'\right]
\end{equation}
then
\begin{equation}
	\dfrac{\dot{C}^\star}{\delta C} \left[1-Q\right] = 
	 D \left(\partial_r s\right)^2 Q''
	+\left( D\left[\left(\partial_{rr}s\right) + \dfrac{\alpha}{r} \left(\partial_r s\right) \right] - \left[ \left(\partial_t s\right) + \dot{R} \left(\dfrac{R}{r}\right)^\alpha \left(\partial_r s\right)\right]\right)Q'
\end{equation}

\subsection{First trial}
If we choose
\begin{equation}
	s = r/R(t),\; r = s R(t),\; \partial_r s = 1/R,\;\partial_{rr} s = 0, \partial_t s = - s \dot{R}/R
\end{equation}
we get
\begin{equation}
\dfrac{\dot{C}^\star}{\delta C} \left[1-Q\right] = \dfrac{D}{R^2} Q'' + 
\left( \dfrac{1}{s} \dfrac{D \alpha}{R^2} - \dfrac{\dot{R}}{R}\left[ \dfrac{1}{s^\alpha} - s \right]\right) Q'
\end{equation}
or
\begin{equation} 
\dfrac{R^2}{D}\dfrac{\dot{C}^\star}{\delta C} \left[1-Q\right] = Q'' + \left( \dfrac{\alpha}{s} + \dfrac{R\dot{R}}{D} \left[s-\dfrac{1}{s^\alpha}\right]\right) Q'
\end{equation}
and using the time
\begin{equation}
	\tau = \dfrac{A}{2\pi D}
\end{equation}
we get
\begin{equation}
	2\dfrac{\dot{C}^\star}{\delta C} \left[1-Q\right] \tau = Q'' + \left( \dfrac{\alpha}{s} + \dot{\tau} \left[s-\dfrac{1}{s^\alpha}\right]\right) Q'
\end{equation}

\subsection{Second trial}
We may also use
\begin{equation}
	s = r^2/R^2,\; r = R\sqrt{s},\; \partial_r s = 2 \sqrt{s}/R = ,\;\partial_{rr} s = 1/R^2, \partial_t s = -2 r^2 \dot{R}/R^3 = -2 s \dot{R}/R
\end{equation}
\end{document}

with $Q(1)=0$ and $Q(\infty)=1$.
We get
\begin{equation}
	\partial_r C = \dfrac{\delta C}{\Lambda} Q',\; \partial_{rr} C = \dfrac{\delta C}{\Lambda^2} Q''
\end{equation}
and
\begin{equation}
	\partial_t C = \left[1-Q\right] \dot{C}^\star - \delta C \dfrac{\dot{\Lambda}}{\Lambda} s Q'.
\end{equation}
The generic equation becomes
\begin{equation}
	\left[1-Q\right] \dot{C}^\star - \delta C \dfrac{\dot{\Lambda}}{\Lambda} s Q' 
	+ \dot{R} \left(\dfrac{R}{s\Lambda}\right)^\alpha \dfrac{\delta C}{\Lambda} Q' = \dfrac{D\delta C}{\Lambda^2}\left( Q'' + \dfrac{\alpha}{s} Q' \right)
\end{equation}
or
\begin{equation}
	\dfrac{\Lambda^2 \dot{C}^\star}{D \delta C} \left[1-Q\right] 
	+ \left[ \Lambda \dot{R} \left(\dfrac{R}{s\Lambda}\right)^\alpha - s \Lambda \dot{\Lambda} \right] \dfrac{Q'}{D}
	= \left( Q'' + \dfrac{\alpha}{s} Q' \right)
\end{equation}
and with $\Lambda(t)=R(t)$,
\begin{equation}
\dfrac{R^2 \dot{C}^\star}{D \delta C} \left[1-Q\right] + \dfrac{R\dot{R}}{D} \left[ \dfrac{1}{s^\alpha} - s \right] Q' =  \left( Q'' + \dfrac{\alpha}{s} Q' \right).
\end{equation}

Using the projected area $A(t) = \pi R^2$, $\dot{A}=2\pi R \dot{R}$, and the time
\begin{equation}
	\tau = \dfrac{A}{2\pi D}
\end{equation}

\begin{equation}
	\dfrac{2 \dot{C}^\star}{\delta C} \left[1-Q\right] \tau + \dot{\tau} \left[ \dfrac{1}{s^\alpha} - s \right] Q' =  \left( Q'' + \dfrac{\alpha}{s} Q' \right).
\end{equation}

\end{document}