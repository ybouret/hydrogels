\documentclass[11pt]{revtex4}
\usepackage{graphicx}
\usepackage{amssymb,amsmath}
\usepackage{epstopdf}
\usepackage{bm}
\DeclareGraphicsRule{.tif}{png}{.png}{`convert #1 `dirname #1`/`basename #1 .tif`.png}


%\date{}                                           % Activate to display a given date or no date



\begin{document}
\title{Let's grow one bubble}

\maketitle

\section{Description}

\subsection{State equation (may change)}
We have $N(t)$ moles of gas in a bubble of radius $R(t)$, and
a dissolved concentration $C(r\geq R,t)$ in the viscoelastic liquid.
The mass evolution law is given by the thermodynamic law of the gaz:
\begin{equation}
	\label{eq:evolve}
	P^\star(t) V(t) = N(t) \mathcal{R}T = N(t) \Theta
\end{equation}

\subsection{Liquid incompressibility}
Let $\vec{v}$ be the velocity field with $\mathrm{div}\,\vec{v}=0$.
We have \begin{equation}
	\label{eq:v2d}
	\vec{v}(r) = \dot{R} \left(\dfrac{R}{r}\right)^\alpha \vec{u}_{r}
\end{equation}
with $\alpha=\dim-1$.

\subsection{Mass exchange}
The mass exchange is in 2D, with a thickness $e$
\begin{equation}
	\label{eq:xch2d}
	\partial_t N = 2\pi R(t) e D \partial_r C(r,t)\vert_{r=R(t)}
\end{equation}
and in 3D
\begin{equation}
	\label{eq:xch3d}
	\partial_t N = 4\pi R^2(t) D \partial_r C(r,t)\vert_{r=R(t)}.
\end{equation}

\subsection{Mass Transport}
The advection-diffusion equation is
\begin{equation}
	\label{eq:transport0}
	\partial_t C + \mathrm{div}(C\vec{v}) = D \Delta C
\end{equation}
which reduces to
\begin{equation}
	\label{eq:transport}
	\partial_t C + \vec{v}\cdot\vec{\nabla}C = D \Delta C.
\end{equation}
We got
\begin{equation}
\label{eq:transport}
\partial_t C + \dot{R} \left(\dfrac{R}{r}\right)^\alpha \partial_r C 
=  D \left(\partial_r^2 C + \dfrac{\alpha}{r} \partial_r C  \right) 
=  D \dfrac{1}{r^\alpha} \partial_r \left(r^\alpha \partial_r C \right)
\end{equation}

%We got in 2D
%\begin{equation}
%	\label{eq:trn2d}
%	\partial_t C + \dot{R} \dfrac{R}{r} \partial_r C = D\left(\partial_r^2 C + \dfrac{1}{r} \partial_r C \right) = D \dfrac{1}{r}\partial_r\left(r\partial_r C\right)
%\end{equation}
%and in 3D
%\begin{equation}
%	\label{eq:trn3d}
%	\partial_t C + \dot{R}\dfrac{R^2}{r^2} \partial_r C = D\left(\partial_r^2 C + \dfrac{2}{r} \partial_r C \right) 
%	= D \dfrac{1}{r^2}\partial_r\left(r^2\partial_r C\right).
%\end{equation}

\section{Coupling}
\subsection{Boundary condition}
We assume that the mechanical equilibrium is reached, so that within the bubble the Henry's law is met
\begin{equation}
	P^\star(t) = k_H C^\star(t)
\end{equation}
and we also get the user's boundary condition
\begin{equation}
	C_\infty
\end{equation}

\subsection{Rescaling}
We will rescale the length by $R(t)$ by using $s=\dfrac{r}{R(t)}$.
We want to solve
\begin{equation}
	\label{eq:trnAll}
	\partial_t C + \dot{R}\left(\dfrac{R}{r}\right)^\alpha \partial_r C = D\left(\partial_r^2 C+ \dfrac{\alpha}{r} \partial_r C \right).
\end{equation}
We are looking for solutions in the form
\begin{equation}
	C(r,t) = C^\star(t) + \left[ C_\infty - C^\star(t)\right] Q(s).
\end{equation}
with $Q(1)=0$ and $Q(\infty)=1$.
We obtain
$$
	\left\lbrace
	\begin{array}{rcl}
	\partial_r C   & = & \dfrac{\delta C(t)}{R(t)}\partial_s Q\\
	\\
	\partial_r^2 C & = & \dfrac{\delta C(t)}{R^2(t)}\partial_s^2 Q\\
	\\
	\partial_t  C  & = & \dot{C^\star}\left[1-Q\right] - s \delta C  \dfrac{\dot{R}}{R} \partial_s Q.\\
	\end{array}
	\right.
$$
Injecting into \eqref{eq:trnAll} we get
$$
	\dot{C^\star}\left[1-Q\right] - s \delta C  \dfrac{\dot{R}}{R} \partial_s Q
	+ \dfrac{\dot{R}}{s^\alpha}  \dfrac{\delta C}{R}\partial_s Q = \dfrac{D\delta C}{R^2}\left[ \partial_s^2 Q + \dfrac{\alpha}{s}\partial_s Q\right]
$$
or
$$
	\dot{C^\star}\left[1-Q\right] + \delta C \dfrac{\dot{R}}{R} \left( \dfrac{1}{s^\alpha} -s \right) \partial_s Q = \dfrac{D\delta C}{R^2}\left[ \partial_s^2 Q + \dfrac{\alpha}{s}\partial_s Q\right]
$$
and finally
\begin{equation}
	\dfrac{R^2}{D} \dfrac{\dot{C^\star}}{\delta C} \left[1-Q\right] + \dfrac{R\dot{R}}{D} \left( \dfrac{1}{s^\alpha} -s \right) \partial_s Q = \left[ \partial_s^2 Q + \dfrac{\alpha}{s}\partial_s Q\right]
\end{equation}

\subsection{Using Experimental Approximation}
\subsubsection{Formal Expression}
For small bubbles, we observe that the projected area growth almost linearly, so that we may discard the first term (TODO: for other reasons too...)
and we get
\begin{equation}
	\underbrace{\dfrac{\dot{A}(t)}{2\pi D}}_{\lambda} \left( \dfrac{1}{s^\alpha} -s \right) \partial_s Q = \left[ \partial_s^2 Q + \dfrac{\alpha}{s}\partial_s Q\right].
\end{equation}

\begin{equation}
	\ln \left(\dfrac{\partial_s Q}{\partial_sQ^\star}\right) = \int_1^s \left[\dfrac{\lambda}{s^\alpha} - \lambda s -\dfrac{\alpha}{s}\right] \,\mathrm{d}s
\end{equation}

In 2D, $\alpha=1$ we get
\begin{equation}
	\label{eq:dsQ2d}
	\partial_s Q_{2d} = \partial_s Q^\star \, e^{\frac{\lambda}{2}}s^{(\lambda-1)}e^{-\frac{\lambda}{2}s^2} = \partial_s Q^\star \, f_{2d}(\lambda,s).
\end{equation}

In 3D, $\alpha=2$, we get
\begin{equation}
	\label{eq:dsQ3d}
	\partial_s Q_{3d} = \partial_s Q^\star \, \dfrac{1}{s^2} e^{\frac{3\lambda}{2}} e^{-\frac{\lambda}{s}} e^{-\frac{\lambda}{2}s^2} = \partial_s Q^\star \, f_{3d}(\lambda,s).
\end{equation}

The slope is obtained by
$$
	\int_1^\infty \partial_s Q \, \mathrm{d} s = 1
$$
leading to
\begin{equation}
	\partial_sQ^\star_{zd} = \left[ \int_1^\infty f_{zd}(\lambda,s) \mathrm{d}s \right] ^{-1}
\end{equation}

\subsubsection{In 2D}
\begin{equation}
	\begin{array}{rl}
	I_2(\lambda) = \int_1^\infty f_2(\lambda,s) \mathrm{d}s & =  e^{\frac{\lambda}{2}}\int_1^\infty s^{(\lambda-1)}e^{-\frac{\lambda}{2}s^2 } \mathrm{d}s\\
	\end{array}
\end{equation}
with 
$$
	I_2(1) = \dfrac{\sqrt{2\pi}}{2} e^{\frac{1}{2}} \mathrm{erfc}
	\left(\dfrac{\sqrt{2}}{2}\right) \approx 0.655679542418798
$$
we get 
$$
	I_2(\lambda) \simeq \dfrac{I_2(1)}{\lambda^{0.374954456663323}}
$$
so that
\begin{equation}
	\partial_s Q^\star_{2d} \simeq 1.525 \times \lambda^{0.375} = \phi_1 \lambda^{\omega_1}
\end{equation}


\subsubsection{In 3D}
\begin{equation}
	I_3(\lambda) = \int_1^\infty \dfrac{1}{s^2} e^{\frac{3\lambda}{2}} e^{-\frac{\lambda}{s}} e^{-\frac{\lambda}{2}s^2} \mathrm{d}s.
\end{equation}
We find
\begin{equation}
	\partial_s Q^\star_{3d} = 2.308 \lambda^{0.288} = \phi_2 \lambda ^ {\omega_2}
\end{equation}

\subsection{Matter Exchange}
\subsubsection{Generic case}
We get
\begin{equation}
	\partial_r C^\star = \dfrac{\delta C(t)}{R(t)} \partial_s Q^\star = \dfrac{\delta C(t)}{R(t)} \phi_\alpha \left(\dfrac{\dot{A}}{2\pi D}\right)^{\omega_\alpha}
\end{equation}

\subsubsection{In 2D}
\begin{equation}	
	\label{eq:xch2d}
	\left\lbrack \Theta \partial_t N\right\rbrack_{2d} 
	= 2\pi R(t) e \Theta D \partial_r C^\star
	=  2\pi e \Theta D \delta C(t) \phi_1 \left(\dfrac{\dot{A}}{2\pi D}\right)^{\omega_1}
\end{equation}
then
\begin{equation}
	\left\lbrack \Theta \partial_t N\right\rbrack_{2d} = \partial_t\left(P^\star V\right) = e \partial_t\left(P^\star A(t)\right)
\end{equation}
so that
\begin{equation}
	\partial_t \left( P^\star \dfrac{A(t)}{2\pi D}\right) = \dfrac{\Theta}{k_H}  \left[P_\infty- P^\star\right] \phi_1 \left(\dfrac{\dot{A}}{2\pi D}\right)^{\omega_1}
\end{equation}
or
\begin{equation}
	\label{eq:drive2d}
	\dfrac{\dot{A}}{2\pi D} = -\dfrac{\dot{P}^\star}{P^\star} \dfrac{A(t)}{2\pi D} + \dfrac{\Theta}{k_H}  \left[\dfrac{P_\infty}{P^\star}-1\right] \phi_1 \left(\dfrac{\dot{A}}{2\pi D}\right)^{\omega_1}
\end{equation}

\subsubsection{In 3D}
\begin{equation}
	\label{eq:xch3d}
	\left\lbrack \Theta \partial_t N\right\rbrack_{3d} =  4\pi R^2(t) \Theta D  \partial_r C^\star = 4\pi R(t) \Theta D  \delta C(t) \phi_2 \left(\dfrac{\dot{A}}{2\pi D}\right)^{\omega_2}.
\end{equation}
then
\begin{equation}
	\left\lbrack \Theta \partial_t N\right\rbrack_{3d} = \partial_t\left(P^\star V\right) = \dfrac{4}{3\sqrt{\pi}} \partial_t \left( P^\star A^{3/2}\right)
\end{equation}
then
\begin{equation}
	\partial_t \left( P^\star A^{3/2}\right) = 3\pi \sqrt{A} \Theta D \delta C  \phi_2 \left(\dfrac{\dot{A}}{2\pi D}\right)^{\omega_2} =
	\dot{P}^\star A^{3/2} + \dfrac{3}{2} P^\star \dot{A} \sqrt{A}
\end{equation}
and finally
\begin{equation}
	\dfrac{\dot{A}}{2\pi D} = -\dfrac{2}{3} \dfrac{\dot{P}^\star}{P^\star} \dfrac{A(t)}{2\pi D} + \dfrac{\Theta}{k_H}  \left[\dfrac{P_\infty}{P^\star}-1\right] \phi_2 \left(\dfrac{\dot{A}}{2\pi D}\right)^{\omega_2}
\end{equation}

\end{document}
