\documentclass[11pt]{revtex4}
\usepackage{graphicx}
\usepackage{amssymb,amsmath}
\usepackage{epstopdf}
\usepackage{bm}
\DeclareGraphicsRule{.tif}{png}{.png}{`convert #1 `dirname #1`/`basename #1 .tif`.png}


%\date{}                                           % Activate to display a given date or no date



\begin{document}
\title{Let's grow one bubble}

\maketitle

\section{Description}
\subsection{Bubble}
We assume that a bubble with a volume $V$ growths with a pressure $P$ and a number of
gaseous mol $N$.
We want to find the growth rate under the assumption that the bubble is formed with dissolved gaz
and that the gas must diffuse through the expanding fluid.\\
Since this can be in 2 or 3 dimensions, we give the bubble a radius $R$.
We set $\alpha=1$ for 2D (and the bubble has a height $e$), and $\alpha=2$ for $3D$.

\subsection{Velocity fluid}
The incompressible fluid velocity is assume to be
\begin{equation}
	\vec{v}(r,t) = \dot{R} \left(\dfrac{R}{r}\right)^\alpha \vec{u}_r
\end{equation}

\subsection{Matter exchange}
In the following, the $\star$ exponent is for the contact value, at $r=R(t)$.\\

\subsubsection{In 2D}
In 2D
\begin{equation}
	\left[\partial_t N\right]_{2d} = 2\pi e R  D \partial_r C^\star.
\end{equation}
giving,  with $\rho=\dfrac{\mathcal{R}T}{k_H}$,
\begin{equation}
	\partial_t\left( C^\star \pi e R^2\right) = \rho 2\pi e R  D \partial_r C^\star.
\end{equation}
then
\begin{equation}
	\dot{C}^\star R + 2 C^\star \dot{R}  = 2\rho  D \partial_r C^\star.
\end{equation}
and finally
\begin{equation}
	\dot{R} = \dfrac{ 2\rho  D \partial_r C^\star - \dot{C}^\star R }{2C^\star}
\end{equation}

\subsubsection{In 3D}
\begin{equation}
	\left[\partial_t N\right]_{3d} =  4\pi R^2 D \partial_r C^\star.
\end{equation}
Then
\begin{equation}
	\partial_t\left( C^\star \dfrac{4\pi}{3} R^3 \right) = \rho   4\pi R^2 D \partial_r C^\star
\end{equation}
or
\begin{equation}
	\partial_t\left( C^\star R^3\right) = 3\rho R^2 D \partial_r C^\star
\end{equation}
and
\begin{equation}
	\dot{C}^\star R + 3 C^\star \dot{R}  = 3\rho  D \partial_r C^\star
\end{equation}
and finally
\begin{equation}
	\dot{R} = \dfrac{3\rho  D \partial_r C^\star - R\dot{C}^\star}{3C^\star}
\end{equation}

\subsubsection{Compact Form}
\begin{equation}
	\dot{R} = \dfrac{(\alpha+1)\rho  D \partial_r C^\star - R\dot{C}^\star}{(\alpha+1)C^\star} = \dfrac{\rho}{C^\star} D\partial_rC^\star - \dfrac{R\dot{C}^\star}{\left(\alpha+1\right)C^\star}
\end{equation}

\subsection{Matter transport}
The incompressible diffusion equation is
\begin{equation}
	\partial_t C + \dot{R} \left(\dfrac{R}{r}\right)^\alpha  \partial_r C = D \left( \partial_{rr} C + \dfrac{\alpha}{r} \partial_r C \right) = D \, \dfrac{1}{r^\alpha} \partial_r \left(r^\alpha\partial_r C\right)
\end{equation}

We use the variable change 
\begin{equation}
	s = r / R(t)
\end{equation}
and we assume that
\begin{equation}
	C(r,t) = C_\infty(t) Q\left(s,t\right)
\end{equation}
We get
\begin{equation}
	\partial_r C = \dfrac{C_\infty}{R} \partial_s Q 
\end{equation}

\begin{equation}
	\partial_{rr} C = \dfrac{C_\infty}{R^2} \partial_{ss} Q 
\end{equation}

\begin{equation}
	\partial_t C  = \dot{C}_\infty Q + C_\infty \left[ -\dfrac{r \dot{R}}{R^2} \partial_s Q + \partial_t Q \right]
\end{equation}

so that

\begin{equation}
	\dot{C}_\infty Q + C_\infty \left[ -\dfrac{r \dot{R}}{R^2} \partial_s Q + \partial_t Q \right] + 
	\dot{R} \left(\dfrac{R}{r}\right)^\alpha \dfrac{C_\infty}{R} \partial_s Q 
	= \dfrac{DC_\infty}{R^2} \left(\partial_{ss}Q + \alpha\dfrac{R}{r} \partial_s Q\right)
\end{equation}
and
\begin{equation}
 \dfrac{R^2}{D} \partial_t Q + \dfrac{R^2 \dot{C}_\infty}{D C_\infty} Q 
 + \dfrac{R\dot{R}}{D}\left[ \dfrac{1}{s^\alpha} -s\right] \partial_s Q
 = \left(\partial_{ss}Q + \dfrac{\alpha}{s} \partial_s Q\right)
\end{equation}

using
\begin{equation}
	\tau = \dfrac{A}{2\pi D} = \dfrac{R^2}{2D}, \dot{\tau} = \dfrac{R\dot{R}}{D}
\end{equation}
we get
\begin{equation}
	2\tau\left[\partial_t Q + \dfrac{\dot{C}_\infty}{C_\infty} Q \right] = \partial_{ss}Q + \left[\dfrac{\alpha}{s} + \dot{\tau}\left(s-\dfrac{1}{s^\alpha}\right) \right] \partial_s Q
\end{equation}

\end{document}
