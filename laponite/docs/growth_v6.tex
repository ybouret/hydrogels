\documentclass[11pt]{revtex4}
\usepackage{graphicx}
\usepackage{amssymb,amsmath}
\usepackage{epstopdf}
\usepackage{bm}
\DeclareGraphicsRule{.tif}{png}{.png}{`convert #1 `dirname #1`/`basename #1 .tif`.png}


%\date{}                                           % Activate to display a given date or no date



\begin{document}
\title{Let's grow one bubble}

\maketitle

\section{Description}
\subsection{Bubble}
We assume that a bubble with a volume $V$ growths with a pressure $P$ and a number of
gaseous mol $N$.
We want to find the growth rate under the assumption that the bubble is formed with dissolved gaz
and that the gas must diffuse through the expanding fluid.\\
Since this can be in 2 or 3 dimensions, we give the bubble a radius $R$.
We set $\alpha=1$ for 2D (and the bubble has a height $e$), and $\alpha=2$ for $3D$.

\subsection{Velocity fluid}
The incompressible fluid velocity is assume to be
\begin{equation}
	\vec{v}(r,t) = \dot{R} \left(\dfrac{R}{r}\right)^\alpha \vec{u}_r
\end{equation}

\subsection{Matter exchange}
In the following, the $\star$ exponent is for the contact value, at $r=R(t)$.\\

\subsubsection{In 2D}
In 2D
\begin{equation}
	\left[\partial_t N\right]_{2d} = 2\pi e R  D \partial_r C^\star.
\end{equation}
giving,  with $\rho=\dfrac{\mathcal{R}T}{k_H}$,
\begin{equation}
	\partial_t\left( C^\star \pi e R^2\right) = \rho 2\pi e R  D \partial_r C^\star.
\end{equation}
then
\begin{equation}
	\dot{C}^\star R + 2 C^\star \dot{R}  = 2\rho  D \partial_r C^\star.
\end{equation}
and finally
\begin{equation}
	\dot{R} = \dfrac{ 2\rho  D \partial_r C^\star - \dot{C}^\star R }{2C^\star}
\end{equation}

\subsubsection{In 3D}
\begin{equation}
	\left[\partial_t N\right]_{3d} =  4\pi R^2 D \partial_r C^\star.
\end{equation}
Then
\begin{equation}
	\partial_t\left( C^\star \dfrac{4\pi}{3} R^3 \right) = \rho   4\pi R^2 D \partial_r C^\star
\end{equation}
or
\begin{equation}
	\partial_t\left( C^\star R^3\right) = 3\rho R^2 D \partial_r C^\star
\end{equation}
and
\begin{equation}
	\dot{C}^\star R + 3 C^\star \dot{R}  = 3\rho  D \partial_r C^\star
\end{equation}
and finally
\begin{equation}
	\dot{R} = \dfrac{3\rho  D \partial_r C^\star - R\dot{C}^\star}{3C^\star}
\end{equation}

\subsubsection{Compact Form}
\begin{equation}
	\dot{R} = \dfrac{(\alpha+1)\rho  D \partial_r C^\star - R\dot{C}^\star}{(\alpha+1)C^\star} = \dfrac{\rho}{C^\star} D\partial_rC^\star - \dfrac{R\dot{C}^\star}{\left(\alpha+1\right)C^\star}
\end{equation}

\subsection{Matter transport}
The incompressible diffusion equation is
\begin{equation}
	\partial_t C + \dot{R} \left(\dfrac{R}{r}\right)^\alpha  \partial_r C = D \left( \partial_{rr} C + \dfrac{\alpha}{r} \partial_r C \right) = D \, \dfrac{1}{r^\alpha} \partial_r \left(r^\alpha\partial_r C\right)
\end{equation}

We use the variable change 
\begin{equation}
	s = r / R(t)
\end{equation}
and
\begin{equation}
	C(r,t) = C_\infty(t) \left[1-Q(s,t)\right]
\end{equation}
so that
\begin{equation}
	\partial_t C = \dot{C}_\infty \left[1-Q\right] - C_\infty \left[ -\dfrac{r\dot{R}}{R^2} \partial_s Q \right]
\end{equation}

\begin{equation}
	\partial_r C = - C_\infty \dfrac{1}{R} \partial_s Q
\end{equation}

\begin{equation}
	\partial_{rr} C = - C_\infty \dfrac{1}{R^2} \partial_{ss} Q
\end{equation}

we get
\begin{equation}
	 \dot{C}_\infty \left[1-Q\right] - C_\infty \left[ -\dfrac{r\dot{R}}{R^2} \partial_s Q \right] 
	 - \dot{R} \left(\dfrac{R}{r}\right)^\alpha  C_\infty \dfrac{1}{R} \partial_s Q =
	  - \dfrac{DC_\infty}{R^2} \left(\partial_{ss} Q + \alpha \dfrac{R}{r} \partial_s Q\right)
\end{equation}
then
\begin{equation}
	\dfrac{R^2}{D}\dfrac{ \dot{C}_\infty}{C_\infty}\left[1-Q\right] + \dfrac{R\dot{R}}{D}\left(s-\dfrac{1}{s^\alpha}\right)\partial_s Q =  - \left(\partial_{ss}Q + \dfrac{\alpha}{s} \partial_s{Q}\right)
\end{equation}

\end{document}


\section{Matter Intake}
\subsection{Generic case}
We use the scaled variable
\begin{equation}
	s(r,t)
\end{equation}
and
\begin{equation}
	C(r,t) = C^\star + \left[C_\infty-C^\star\right] Q(s) = C^\star + \delta C Q(s)
\end{equation}
so that
\begin{equation}
	\partial_r C = \delta C \left(\partial_r s\right) Q'(s)
\end{equation}
and
\begin{equation}
	\partial_{rr} C = \delta C \left[ \left(\partial_r s\right)^2 Q'' + \left(\partial_{rr}s\right) Q'\right]
\end{equation}
and
\begin{equation}
	\partial_{t} C = \dot{C}^\star \left[1-Q\right] + \delta C \left(\partial_t s\right) Q'
\end{equation}

We obtain
\begin{equation}
	\dfrac{\dot{C}^\star}{\delta C} \left[1-Q\right] +  \left(\partial_t s\right) Q' 
	+\dot{R} \left(\dfrac{R}{r}\right)^\alpha \left(\partial_r s\right) Q' = D  \left[ \left(\partial_r s\right)^2 Q'' + \left(\partial_{rr}s\right) Q' + \dfrac{\alpha}{r} \left(\partial_r s\right) Q'\right]
\end{equation}
then
\begin{equation}
	\dfrac{\dot{C}^\star}{\delta C} \left[1-Q\right] = 
	 D \left(\partial_r s\right)^2 Q''
	+\left( D\left[\left(\partial_{rr}s\right) + \dfrac{\alpha}{r} \left(\partial_r s\right) \right] - \left[ \left(\partial_t s\right) + \dot{R} \left(\dfrac{R}{r}\right)^\alpha \left(\partial_r s\right)\right]\right)Q'
\end{equation}

\subsection{Substitution}
If we choose
\begin{equation}
	s = r/R(t),\; r = s R(t),\; \partial_r s = 1/R,\;\partial_{rr} s = 0, \partial_t s = - s \dot{R}/R
\end{equation}
we get
\begin{equation}
\dfrac{\dot{C}^\star}{\delta C} \left[1-Q\right] = \dfrac{D}{R^2} Q'' + 
\left( \dfrac{1}{s} \dfrac{D \alpha}{R^2} - \dfrac{\dot{R}}{R}\left[ \dfrac{1}{s^\alpha} - s \right]\right) Q'
\end{equation}
or
\begin{equation} 
\dfrac{R^2}{D}\dfrac{\dot{C}^\star}{\delta C} \left[1-Q\right] = Q'' + \left( \dfrac{\alpha}{s} + \dfrac{R\dot{R}}{D} \left[s-\dfrac{1}{s^\alpha}\right]\right) Q'
\end{equation}
and using the time
\begin{equation}
	\tau = \dfrac{A}{2\pi D} = \dfrac{R^2}{2D}, \;\; \dot{\tau} = \dfrac{\dot{A}}{2\pi D} = \dfrac{R\dot{R}}{D}
\end{equation}
we get
\begin{equation}
	2\dfrac{\dot{C}^\star}{\delta C} \left[1-Q\right] \tau = Q'' + \left( \dfrac{\alpha}{s} + \dot{\tau} \left[s-\dfrac{1}{s^\alpha}\right]\right) Q'
\end{equation}

Moreover, we have
\begin{equation}
	\partial_r C^\star = \dfrac{\delta C}{R} Q'^\star
\end{equation}
then
\begin{equation}
	R \dot{R} = \rho \dfrac{\delta C}{C^\star} D Q'^\star - \dfrac{R^2\dot{C}^\star}{\left(1+\alpha\right)C^\star}
\end{equation}
and
\begin{equation}
\dot{\tau} = \rho \dfrac{\delta C}{C^\star} Q'^\star - \dfrac{2}{1+\alpha} \dfrac{\dot{C}^\star}{C^\star} \tau
\end{equation}

\subsection{Some solutions}

\subsubsection{Constant inside conc}
We get
\begin{equation}
 0 = Q'' + \left( \dfrac{\alpha}{s} + \dot{\tau} \left[s-\dfrac{1}{s^\alpha}\right]\right) Q'
\end{equation}
Using $Y=Q'$,
\begin{equation}
	0 = Y' + \left( \dfrac{\alpha}{s} + \dot{\tau} \left[s-\dfrac{1}{s^\alpha}\right]\right) Y
\end{equation}
so that
\begin{equation}
	\log Y - \log Y^\star = -\int_{1}^s \left( \dfrac{\alpha}{u} + \dot{\tau} \left[u-\dfrac{1}{u^\alpha}\right]\right) \, \mathrm{d}u
\end{equation}
We get
\begin{equation}
\left\lbrace
\begin{array}{rclcl}
\log Y_1 - \log Y^\star_1 & = &\int_{1}^s \left( \dfrac{\dot{\tau}-1}{u} - \dot{\tau} u \right) \, \mathrm{d}u & = & (\dot{\tau}-1) \log s - \dfrac{\dot{\tau}}{2}(s^2-1)\\
\\
\log Y_2 - \log Y^\star_2 & = &\int_{1}^s \left( \dfrac{\dot{\tau}}{u^2} - \dfrac{2}{u} - \dot{\tau} u \right) \, \mathrm{d}u & = & \dot{\tau}\left(1-\dfrac{1}{s}\right)-2 \log s - \dfrac{\dot{\tau}}{2}(s^2-1)\\
\end{array}
\right.
\end{equation}
then
\begin{equation}
\left\lbrace
\begin{array}{rcl}
Y_1(s) & = & Y^\star_1 e^{\frac{\dot{\tau}}{2}} s^{(\dot{\tau}{-1})} e^{-\frac{\dot{\tau}}{2}s^2}\\
\\
Y_2(s) & = & Y^\star_2 \dfrac{1}{s^2} e^{\frac{3\dot{\tau}}{2}} e^{-\frac{\dot{\tau}}{s}} e^{-\frac{\dot{\tau}}{2}s^2}\\
\end{array}
\right.
\end{equation}

We get
\begin{equation}
	Y_1^\star = \dfrac{
		2e^{-\dfrac{\dot{\tau}}{2}}
		\left(\dfrac{\dot{\tau}}{2}\right)^{\dfrac{\dot{\tau}}{2}}
	}
	{\Gamma\left(\dfrac{\dot{\tau}}{2}\right)-\gamma\left(\dfrac{\dot{\tau}}{2},\dfrac{\dot{\tau}}{2}\right)}
\end{equation}



\subsubsection{Other?}
We look for solution
\begin{equation}
	Q(s) = 1 - e^{-f(s)}
\end{equation}
leading to
\begin{equation}
	2\tau \dfrac{\dot{C}^\star}{\delta C} = f''(s) - f'^2(s) + \left( \dfrac{\alpha}{s} + \dot{\tau} \left[s-\dfrac{1}{s^\alpha}\right]\right) f'(s)
\end{equation}

\section{Try Again}

\subsection{Equation and rescaling}
The incompressible diffusion equation is
\begin{equation}
	\partial_t C + \dot{R} \left(\dfrac{R}{r}\right)^\alpha  \partial_r C = D \left( \partial_{rr} C + \dfrac{\alpha}{r} \partial_r C \right) = D \, \dfrac{1}{r^\alpha} \partial_r \left(r^\alpha\partial_r C\right)
\end{equation}
and
with 
\begin{equation}
	s = r/R(t)
\end{equation}
and
\begin{equation}
	C(r,t) = C^\star + \left[C_\infty-C^\star\right] Q(s,t) = C^\star + \delta C Q(s,t)
\end{equation}
we get
\begin{equation}
	\partial_r C = \dfrac{\delta C}{R} \partial_s Q, \; \partial_{rr} C = \dfrac{\delta C}{R^2} \partial_s Q
\end{equation}
and
\begin{equation}
	\partial_t C = \dot{C}^\star \left[1-Q\right] + \delta C \left[\partial_t Q - s \dfrac{\dot{R}}{R} \partial_s Q \right]
\end{equation}
We get
\begin{equation}
	\dot{C}^\star \left[1-Q\right] + \delta C \left[\partial_t Q - s \dfrac{\dot{R}}{R} \partial_s Q \right] + \dfrac{\dot{R}}{R} \dfrac{1}{s^\alpha} \delta C \partial_s Q 
	= \dfrac{D \delta C}{R^2} \left( \partial_{ss} Q + \dfrac{\alpha}{s} \partial_s Q \right)
\end{equation}
and
\begin{equation}
  \dfrac{R^2\dot{C}^\star}{D\delta C}\left[1-Q\right] + \dfrac{R^2}{D} \partial_t Q + \dfrac{R\dot{R}}{D} \left(\dfrac{1}{s^\alpha}-s\right) \partial_s Q =   \partial_{ss} Q + \dfrac{\alpha}{s} \partial_s Q
\end{equation}
then
\begin{equation}
	2 \tau \left( \partial_t Q + \dfrac{\dot{C}^\star}{\delta C} \left[1-Q\right] \right) =
	 \partial_{ss} Q + \left( \dfrac{\alpha}{s} + \dot{\tau} \left[s - \dfrac{1}{s^\alpha} \right] \right) \partial_s Q
\end{equation}

With
\begin{equation}
	Y = 1-Q
\end{equation}
and
\begin{equation}
	\Omega(t) = \dfrac{\dot{C}^\star}{\delta C}
\end{equation}

we get
\begin{equation}
	2 \tau \left( \partial_t Y -  \Omega Y \right) =
	 \partial_{ss} Y + \left( \dfrac{\alpha}{s} + \dot{\tau} \left[s - \dfrac{1}{s^\alpha} \right] \right) \partial_s Y
\end{equation}
Using
\begin{equation}
Y_0(t) = \exp \left[ \int_{t_0}^t \Omega(u) \,\mathrm{d}u\right]
\end{equation}
and
\begin{equation}
Y(s,t) = \Xi(s,t) Y_0(t),
\end{equation}
we get
\begin{equation}
	\tau \partial_t \Xi =  \partial_{ss} \Xi + \left( \dfrac{\alpha}{s} + \dot{\tau} \left[s - \dfrac{1}{s^\alpha} \right] \right) \partial_s \Xi
\end{equation}

\end{document}

\subsection{Exponential Form}
We may look a solution like
\begin{equation}
	Q(s,t) = 1 - e^{-\Xi(s,t)}
\end{equation}
and we get, using
$$
	\Omega(t) = \dfrac{\dot{C}^\star}{\delta C}
$$
\begin{equation}
	2\tau \left( \partial_t \Xi + \Omega \right) = \partial_{ss} \Xi - \left(\partial_s \Xi\right)^2 + \underbrace{\left( \dfrac{\alpha}{s} + \dot{\tau} \left[s - \dfrac{1}{s^\alpha} \right] \right)}_{-\lambda(s,t)} \partial_s \Xi 
\end{equation}

\subsection{Quasi-stat}
With $ \partial_t \Xi \simeq 0$, $\dot{\tau}$ approximately constant, and $Y=\partial_s \Xi$, we get
\begin{equation}
	2\tau\Omega = Y' - Y^2 - \lambda(s) Y 
\end{equation} 
which is a kind of Ricatti equation.

\subsubsection{Kernel}
We solve the kernel equation
\begin{equation}
	Y'_0 = Y^2_0 + \lambda(s)Y_0
\end{equation}
using 
$$
	W = 1/Y_0
$$
so that
\begin{equation}
	-\dfrac{W'}{W^2} = \dfrac{1}{W^2} + \dfrac{\lambda(s)}{W}
\end{equation}
or
\begin{equation}
	W' = -\lambda(s) W + 1
\end{equation}
We define
\begin{equation}
	\Lambda(s) = \int_{1}^s \lambda(u) \, \mathrm{d}u
\end{equation}
and get
\begin{equation}
	W(s) = \left\lbrack W^\star - \int_1^s e^{\Lambda(u)} \, \mathrm{d} u \right\rbrack e^{-\Lambda(s)}
\end{equation}
or
\begin{equation}
	Y_0(s) = \partial_s \Xi = \dfrac{e^{\Lambda(s)}}{ \frac{1}{Y_0^\star} - \int_1^s e^{\Lambda(u)} \, \mathrm{d} u }
\end{equation}
which imposes
\begin{equation}
	Y_0^\star = \left[ 
	\int_1^{+\infty} e^{\Lambda(u)} \, \mathrm{d} u 
	\right]^{-1}
\end{equation}

We have
\begin{equation}
\left\lbrace
\begin{array}{rclcl}
\Lambda_1(s) & = &\int_{1}^s \left( \dfrac{\dot{\tau}-1}{u} - \dot{\tau} u \right) \, \mathrm{d}u & = & (\dot{\tau}-1) \log s - \dfrac{\dot{\tau}}{2}(s^2-1)\\
\\
\Lambda_2(s) & = &\int_{1}^s \left( \dfrac{\dot{\tau}}{u^2} - \dfrac{2}{u} - \dot{\tau} u \right) \, \mathrm{d}u & = & \dot{\tau}\left(1-\dfrac{1}{s}\right)-2 \log s - \dfrac{\dot{\tau}}{2}(s^2-1)\\
\end{array}
\right.
\end{equation}
then
\begin{equation}
\left\lbrace
\begin{array}{rcl}
e^{\Lambda_1(s)} & = & e^{\frac{\dot{\tau}}{2}}s^{\left(\dot{\tau}-1\right)} e^{-\frac{\dot{\tau}}{2}s^2 }\\
\\
e^{\Lambda_2(s)} & = & \dfrac{1}{s^2}e^{\frac{3\dot{\tau}}{2}} e^{-\frac{\dot{\tau}}{s}} e^{-\frac{\dot{\tau}}{2}s^2 }\\
\end{array}
\right.
\end{equation}

\subsubsection{First Order}

We write $Y=Y_0+Y_1$ and we obtain
\begin{equation}
	Y_1' = \Psi + \left[ 2 Y_0(s) + \lambda(s) \right] Y_1(s), \;\; \Psi = 2 \tau \Omega
\end{equation}
We define
\begin{equation}
	\Upsilon(s) = \int_1^s 2Y_0(u) \mathrm{d} u
\end{equation}
we get
\begin{equation}
	Y_1 = \left\lbrack Y_1^\star + \Psi \int_1^s e^{-\Upsilon(u)} e^{-\Lambda(u)} \, \mathrm{d} u  \right\rbrack e^{\Upsilon(s)} e^{\Lambda(s)}
\end{equation}

\end{document}


\end{document}