\documentclass{revtex4}

\usepackage{graphicx}
\usepackage{amssymb,amsmath}
\usepackage{epstopdf}
\usepackage{bm}
\DeclareGraphicsRule{.tif}{png}{.png}{`convert #1 `dirname #1`/`basename #1 .tif`.png}


\newcommand{\trn}{\!\!~^t}

\begin{document}
	\title{PNP}
	\maketitle
	
\section{Notations}
\subsection{PNP Equation}
Let us assume that we have $M$ species $A_1,\ldots,A_M$ in solution.
The electrochemical potential of a species is
\begin{equation}
	\mu_j = \mu^{0}_j + RT \ln \dfrac{\gamma_jC_j}{C_0} + z_j\mathcal{F}\Psi
\end{equation}
where $\gamma_j$ is the activity, $\Psi$ is the electric potential, $C_0=1M$, $z_j$ is the algebraic charge.
The flux is
\begin{equation}
	\vec{F}_j = -D_j 
	\left( 
		\dfrac{1}{\gamma_j} \vec{\nabla} \left( \gamma_j C_j \right) + 
		C_j \dfrac{z_j\mathcal{F}}{RT} \vec{\nabla}\Psi 
	\right)
	=
	 -D_j 
	\left( 
		\vec{\nabla}  C_j  + 
		C_j \dfrac{z_j\mathcal{F}}{RT} \vec{\nabla}\Psi + C_j \vec{\nabla} \ln \gamma_j
	\right)
\end{equation}
or
\begin{equation}
	\vec{F}_j = - \underbrace{D_j\left( \vec{\nabla} C_j + C_j \vec{\nabla} \ln \gamma_j \right)}_{\vec{J}_j} + C_j z_j u_j \vec{E}
\end{equation}
where $u_j = \mathcal{F}D_j/(RT)$ is the absolute ionic mobility.
If we now have $N$ reactions with individual chemical extents $\xi_i$ and
a chemical (univocal) topology matrix $\pmb{\nu}$, plus an extern source term $\vec{\rho}$, the PNP equation is
\begin{equation}
	\partial_t C_j + \mathrm{div}\vec{F}_j = \left(\trn\pmb{\nu} \partial_t \vec{\xi} + \vec{\rho}\right)_j
\end{equation}

\subsection{Generic Poisson Equation}
We shall always have
\begin{equation}
	\Delta \Psi = - \alpha \left(C_{xs}+\sum z_j C_j\right)
\end{equation}
with
$$
	\alpha = \dfrac{10^3\mathcal{F}}{\epsilon}
$$
and $\epsilon_0=8.85\cdot 10^{-12} F.m^{-1}$.

The Bjerrum length is around 1 nm in water: the diffusion can't produce charge separation.
But the Debye length is around a few nanometers: the charge can't interact above that distance...

%\section{Neutral diffusion without external source term}
%A valid topology matrix conserves the charge...
%How to keep the electroneutrality in that case ?
%The standard approach is to cancel the net electric current with an internal electrical field.
%We use
%$$
%	\sum z_j \vec{F}_j = \vec{0}
%$$
\section{Neutralising Field}
We could compute the potential, but we will have some very stiff fields.
Can we write a neutralising field 

\section{Poisson Boltzmann}
\subsection{Quasi-Exact Formulation}
The idea is that the electric field forces the charges to be in an electrochemical steady
state, so that on a domain
\begin{equation}
	C_j(\vec{r}) \propto \tilde{C}_j e^{-\dfrac{z_j\mathcal{F}\Psi(\vec{r})}{RT}}
\end{equation}
and the mass conservation over the domain leads to
\begin{equation}
	\int_V  C_j(\vec{r}) \mathrm{d}\vec{r} = n_j = \tilde{C}_j \int_V e^{-\dfrac{z_j\mathcal{F}\Psi(\vec{r})}{RT}} \mathrm{d}\vec{r}.
\end{equation}
So $\tilde{C}_j$ is an "averaged" concentration
$$
	\tilde{C}_j = \dfrac{n_j}{\int_V e^{-\dfrac{z_j\mathcal{F}\Psi(\vec{r})}{RT}} \mathrm{d}\vec{r}}.
$$
\subsection{Order 1 approximation}
We get
\begin{equation}
	\Delta \Psi 
	%-\alpha \left[ C_{xs} + \sum_j z_j C_j \right]
	%\simeq -\alpha \left[  C_{xs} + \sum_j  \dfrac{z_jn_j}{V}\left(1-\dfrac{z_j\mathcal{F}\Psi}{RT}\right) \right]
	\simeq -\alpha \left[ 	C_{xs} 
							+\left(\sum_j{z_j\bar{C}_j}\right)
							- \dfrac{\mathcal{F}}{RT}\left(\sum_j z_j^2\bar{C}_j\right) \Psi
							\right]
\end{equation}

\end{document}