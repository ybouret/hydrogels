\documentclass{revtex4}

\usepackage{graphicx}
\usepackage{amssymb,amsmath}
\usepackage{epstopdf}
\usepackage{bm}
\DeclareGraphicsRule{.tif}{png}{.png}{`convert #1 `dirname #1`/`basename #1 .tif`.png}


\newcommand{\trn}{\!\!~^t}

\begin{document}
	\title{PNP}
	\maketitle
	
\section{Notations}
Let us assume that we have $M$ species $A_1,\ldots,A_M$ in solution.
The electrochemical potential of a species is
\begin{equation}
	\mu_j = \mu^{0}_j + RT \ln \dfrac{\gamma_j[A_j]}{C_0} + z_j\mathcal{F}\Psi
\end{equation}
where $\gamma_j$ is the activity, $\Psi$ is the electric potential, $C_0=1M$, $z_j$ is the algebraic charge.
The flux is
\begin{equation}
	\vec{F}_j = -D_j 
	\left( 
		\dfrac{1}{\gamma_j} \vec{\nabla} \left( \gamma_j [A_j] \right) + 
		[A_j] \dfrac{z_j\mathcal{F}}{RT} \vec{\nabla}\Psi
	\right)
\end{equation}
or
\begin{equation}
	\vec{F}_j = - \dfrac{D_j}{\gamma_j} \vec{\nabla} \left( \gamma_j [A_j] \right) + [A_j] z_j u_j \vec{E}
\end{equation}
where $u_j = \mathcal{F}D_j/(RT)$ is the absolute ionic mobility.
If we now have $N$ reactions with individual chemical extents $\xi_i$ and
a chemical (univocal) topology matrix $\pmb{\nu}$, plus an extern source term $\vec{\rho}$, the PNP equation is
\begin{equation}
	\partial_t [A_j] + \mathrm{div}\vec{F}_j = \left(\trn\pmb{\nu} \partial_t \vec{\xi} + \vec{\rho}\right)_j
\end{equation}

\section{Generic Poisson Equation}
We shall always have
\begin{equation}
	\Delta \Psi = - \alpha\sum z_j C_j
\end{equation}
with
$$
	\alpha = \dfrac{10^3\mathcal{F}}{\epsilon}
$$

\section{Neutral diffusion without external source term}
A valid topology matrix conserves the charge...
How to keep the electroneutrality in that case ?
The standard approach is to cancel the net electric current with an internal electrical field.
We use
$$
	\sum z_j \vec{F}_j = \vec{0}
$$
\end{document}